\documentclass{article}
\usepackage[utf8]{inputenc}
\usepackage{amsfonts}
\usepackage{soul}
\usepackage{textcomp}
\usepackage{color}
\usepackage{enumitem}
\usepackage{amsmath}
\usepackage{amssymb}
\usepackage{mathtools}
\usepackage{listings}
\usepackage{amsthm}
\definecolor{dkgreen}{rgb}{0,0.6,0}
\definecolor{gray}{rgb}{0.5,0.5,0.5}
\definecolor{mauve}{rgb}{0.58,0,0.82}
\lstset{frame=tb,
    language=Bash,
    aboveskip=3mm,
    belowskip=3mm,
    showstringspaces=false,
    columns=flexible,
    basicstyle={\small\ttfamily},
    numbers=none,
    numberstyle=\tiny\color{gray},
    keywordstyle=\color{blue},
    commentstyle=\color{dkgreen},
    stringstyle=\color{mauve},
    breaklines=true,
    breakatwhitespace=true,
    escapeinside={(*}{*)},          % if you want to add LaTeX within your code
    tabsize=4
}

\title{Notes from Semantics and verification of programs}
\author{Jacek Olczyk}
\date{October 2018}

\begin{document}
\maketitle
\part{Notes from tutorials by Lorenzo Clemente}
\section{Small step semantics - continuation}
\subsection{Recap}
\begin{itemize}
	\item Global environments $ \rho \vdash e \rightarrow e' $
	\item $\frac{\rho[x\rightarrow n]\vdash e \rightarrow e' }{
		 \rho\vdash\text{ let } x= \underline{n}\text{ in }e \rightarrow\text{ let }x = \underline{n}\text{ in }e'}$
\end{itemize}
\subsection{Local environments}
\begin{itemize}
	\item How do we define the semantics for 'let $x$ = $e$ in $f$' expressions using local environments? More precisely, we need $e$ to have its own environment, so that its evaluation doesn't affect the environment of $f$, as is the case with global environments.
	\item We are given the following 2 rules:
	\item $\frac{}{ (\rho, x)\rightarrow(\rho, \rho(x)) }$
	\item $\frac{}{(\rho, \text{ let }x = \underline{n}\text{ in }e)\rightarrow(\rho[x\rightarrow n], e) }$
	\item Now we need to give a rule for evaluating let expressions where a non-numeric expression is assigned to $x$.
	\item $\frac{ (\rho, e)\rightarrow(\rho', e') }
	{ (\rho, \text{ let }x = e\text{ in }f)\rightarrow((\rho?\text{ or maybe }\rho'?), \text{ let } x = e'\text{ in }f) }$
	\item $ \rho $ doesn't work, because then a nested let in expression can't change the value of their variables.
	\item Neither does $ \rho' $, because then we don't get our original environment back at the end.
	\item Solution: new construct
	\item $ e \text{ then }x=n $
	\item Now we have:
	\item $\frac{ (\rho, e)\rightarrow(\rho', e') }
	{ (\rho, e\text{ then }x = \underline{n})\rightarrow(\rho', e'\text{ then } x=\underline{n}) }$
	\item $\frac{}{ (\rho, \underline m\text{ then }x=\underline{n})\rightarrow(\rho[x\rightarrow \underline{n}], \underline{m}) }$
	\item $\frac{}{(\rho, \text{ let }x = \underline{n}\text{ in }e)\rightarrow(\rho[x\rightarrow \underline{n}], e\text{ then }x=\rho(x)) }$
\end{itemize}
\section{Imperative language}

\paragraph{Syntax}
$$C::=\text{ Skip }| X:=e|C;C|\text{if }b\text{ then }c\text{ else }c |\text{while }b\text{ do }c$$
$$e::= n|x|e+e$$
$$b::true|false|e\leq e|\neg b | b\wedge b$$
$$ E[[e]]_{s}\in\mathbb{Q} , B[[b]]_{s}\in\{true, false\}$$
$$ s\in State = Var\rightarrow \mathbb{Q} $$
Configurations
$$ (c, s)\in C $$
$$ s\in C\text{ (final)}$$
Small step rules for C - expressions
$$\frac{}{(Skip, s)\rightarrow s}$$
$$\frac{}{(x:=e, s)\rightarrow s[x\rightarrow E[[e]]_{s}}$$
$$\frac{(c, s)\rightarrow s'}{(c;d, s)\rightarrow (d, s')}$$
$$\frac{(c, s)\rightarrow (c', s')}{(c;d, s)\rightarrow (c';d, s')}$$
$$\frac{B[[b]]_{s}=true}{(\text{if }b\text{ then }c\text{ else }d , s)\rightarrow (c, s)}$$
$$\frac{B[[b]]_{s}=false}{(\text{if }b\text{ then }c\text{ else }d , s)\rightarrow (d, s)}$$
$$\frac{B[[b]]_{s}=true}{(\text{while }b\text{ do }c, s)\rightarrow (c; \text{while }b\text{ do }c, s)}$$
$$\frac{B[[b]]_{s}=false}{(\text{while }b\text{ do }c , s)\rightarrow s}$$
Adding "Repeat c until b"
$$\frac{}{(\text{Repeat }c\text{ until }b, s)\rightarrow (c ; \text{if }b\text{ then }Skip\text{ else }\text{Repeat }c\text{ until }b, s)}$$

\section{Numbers as strings of bits}
\begin{itemize}
	\item Evaluate:
	\item $n::=\$0|\$1|n 0|n 1|n + n$
	\item final configurations: numbers without "+", e.g. \$100101
	\item $ n\rightarrow n' $
	\item $\frac{n\rightarrow n'}{n0\rightarrow n'0}$
	\item $\frac{n\rightarrow n'}{n1\rightarrow n'1}$
	\item $\frac{m\rightarrow m'}{m+n\rightarrow m'+n}$
	\item $\frac{n\rightarrow n'}{m+n\rightarrow m+n'}$
	\item $\frac{}{m0+n0\rightarrow (m+n)0}$
	\item $\frac{}{m0+n1\rightarrow (m+n)1}$
	\item $\frac{}{m1+n0\rightarrow (m+n)1}$
	\item $\frac{}{m1+n1\rightarrow (m+n+\$1)0}$
	\item Fill in the last 4
	\item I think we should add a rule to merge two doll
\end{itemize}
\section{Next time}
Add to the syntax:
\begin{itemize}
	\item for x:=e to e do c 
	\item do e times c
	\item do c while e
\end{itemize}
\section{TODO przepisanie z zeszytu}
\section{Loop, continue and break}
$$C::=\ldots | loop\ c | continue | break$$
Last time we did small steps semantics using $c\ then\ d$ statements. Now we want big steps:
$$\frac{c,s\rightarrow\ldots}{loop\ c,s\rightarrow}\frac{}{continue,s\rightarrow}\frac{}{break,s\rightarrow\ldots}$$
We can change the set of configurations by adding to the existing set of final configurations pairs $(state, flag)$ where $flag\in \{CNT, BRK\}$, thus:
$$\frac{c,s\rightarrow s', loop\ c, s'\rightarrow s''}{loop\ c,s\rightarrow s''}
\frac{c,s\rightarrow (s', CNT), loop\ c, s'\rightarrow s''}{loop\ c,s\rightarrow s''}
\frac{c,s\rightarrow (s', BRK)}{loop\ c,s\rightarrow s'}$$
$$\frac{c,s\rightarrow s',(d,s')\rightarrow s'' }{c;d,s\rightarrow s''}, \hat s\in\{s'', (s'',CNT), (s'',BRK)\}
\frac{c,s\rightarrow (s', f)}{c;d,s\rightarrow (s', f)}
$$
\section{Expressions with side effects}
The syntax is as follows:
$$C::=Skip | x:=e | c;c$$$$e::=x|n|e+e|c\ resultis\ e$$
Old rules:
$$\frac{}{n,s\rightarrow \underline{n}}$$
$$\frac{}{x,s\rightarrow s(x)}$$
$$\frac{e,s\rightarrow \underline{m}\ \ f,s\rightarrow \underline{n}}{e+f,s\rightarrow \underline{\underline{m+n}}}$$
$$\frac{}{Skip,s\rightarrow s}$$
$$\frac{e,s\rightarrow \underline{n}}{x:=e,s\rightarrow s[x\mapsto n]}$$
New rules:
$$\frac{c,s\rightarrow s'\ \ e,s'\rightarrow \underline{n}}{c\ resultis\ e,s\rightarrow \underline{n}}$$
Buut, this doesn't propagate the state change from inside the expressions! To fix this, we change the meaning of $ \rightarrow $ for expressions by making it go to a pair $ (number, state)$. Here are the modified old rules for addition and $resultis$:
$$\frac{e,s\rightarrow \underline{m}, s'\ \ f,s\rightarrow \underline{n}, s''}{e+f,s\rightarrow \underline{\underline{m+n}}, s''}$$
$$\frac{c,s\rightarrow s'\ \ e,s'\rightarrow \underline{n}, s''}{c\ resultis\ e,s\rightarrow \underline{n}, s''}$$

\section{Let in expressions with lazy evaluation}
Previously we had 'call by value' semantics for let in expressions, now we want 'call by name' semantics, which evaluate the variable assignment only when its value is needed. In CBV, we had $\frac{}{s\vDash x\rightarrow s(x)}$. How do we write semantics for $ let $ in CBN?
$$\frac{s[x\mapsto e]\vDash f\rightarrow m}{s\vDash let\ x = e\ in\ f\rightarrow m}$$
$$\frac{s\vDash s(x)\rightarrow n}{s\vDash x\rightarrow \underline{n})}$$
But this is dynamic binding, the environment used is whatever was at the moment of evaluation.
To get static binding, we need variables to record state alongside the expressions: $ St=Var\rightarrow(Expr\times St)\cup \mathbb{Q} $. But this is not a definition, just a recursive equation! Thus, let $St_0=\emptyset$ and $ St_{i+1} = Var\rightarrow(E\times St_i \cup \mathbb{Q}) $. And the whole state is defined like this: (A set of russian dolls with arbitrary nesting) $ St=\bigcup_{i=0}^\infty St_i $.
Now we can get to the rules with static binding:
$$\frac{s(x)=(e,s')\ \ s'\vDash e\rightarrow n}{s\vDash x\rightarrow \underline{n}}$$
$$\frac{s[x\mapsto e, s]\vDash f\rightarrow m}{s\vDash let\ x = e\ in\ f\rightarrow m}$$

\section{Tutorial 14/11}
\subsection{Eager vs. lazy, dynamic vs. static}
\subsubsection{Higher order expression}
$$e::=x|n|e+e|let\ x=e\ in\ e | \lambda x.e|e\ e$$
Where $ \lambda x.e $ is $\lambda$ abstraction - function definition, and $e\ e$ is function application.
Now we find that the $let\ in$ construct is redundant. How do we express its semantics using $ \lambda $ abstraction and application?
$$let\ x = e\ in\ f \equiv (\lambda x.f) e$$
We need to use parentheses because application has the highest priority of all expressions.
\subsubsection{Call-by-value (eager) big step operational semantics.}
Is there a difference between static and dynamic binding in this case?
Without higher order expressions, we can't do dynamic binding, because we have no concept of expressions inside state. 
\paragraph{Is static = dynamic in higher order?}
$$let\ x=7\ in \ let \ f=\lambda y.y+x\ in\ let\ x=3\ in \ f\ 10$$
If we evaluate this expression with static binding, it evaluates to 17, as $ x $ gets mapped inside $ f $ to its value at the time of binding, and with dynamic binding it's 13, because $ x $ is bound to 3 at the time of application of $ f $.
To write the semantics, we introduce closure. For example, $ \lambda y.y+x $ in state $ s $ evaluates to the triplet called closure $ (y, y+x, s) \in Var\times Expr\times St$
\paragraph{Static binding with eager evaluation}
$$
\begin{matrix}
& \text{STATIC}& \text{DYNAMIC}\\
\text{EAGER}&
	\begin{matrix}
	Val=\mathbb{Z}\cup Var\times Expr\times St\\
	St=Var\rightarrow Val\\
	\text{mutually recursive}
	\end{matrix}&\\
\text{LAZY}&&
\end{matrix}
$$
Is it possible to construct sets that satisfy this recursive definition? We'll construct a family of sets for both $ Val $ and $ State $ and define them as infinite unions of all sequences.
$$
\begin{matrix}
Val_0=\emptyset & Val_{n+1}=\mathbb{Z}\cup Var\times Expr\times St_{n+1}\\
St_0=\emptyset  & St_{n+1}=Var\rightarrow Val_n\\
\end{matrix}
$$
Thus, $ Val_1=\mathbb{Z}, Val_2=\mathbb{Z}\cup Var\times Expr\times(Var\rightarrow \mathbb{Z})\ldots $
Now the big step semantics:
$$\frac{}{n,s\rightarrow n}\frac{}{x,s\rightarrow s(x)\in Val}
\frac{e,s\rightarrow\underline{m}, f,s\rightarrow \underline{n}}{e+f,s\rightarrow m+n}$$
And new ones:
$$\frac{}{\lambda x.e,s\rightarrow (x, e,s)},
\frac{(e, s)\rightarrow (x, e', s')\ (f, s) \rightarrow v\ (e', s'[x\mapsto v])\rightarrow v'}{e\ f,s\rightarrow v'}$$
Important: since elements of $Val$ can be either numbers or closures, then effects of our function applications can also be closures!
\paragraph{Dynamic binding with eager evaluation.}
$$
\begin{matrix}
& \text{STATIC}& \text{DYNAMIC}\\
	\text{EAGER} &
	\begin{matrix}
		Val=\mathbb{Z}\cup (Var\times Expr\times St)\\
		St=Var\rightarrow Val\\
		\text{mutually recursive}
	\end{matrix} &
	\begin{matrix}
		Val=\mathbb{Z}\cup (Var\times Expr)\\
		St=Var\rightarrow Val\\
		\text{not recursive anymore!}
	\end{matrix}\\
\text{LAZY}&&
\end{matrix}
$$
Now the rules:
$$\frac{}{\lambda x.e,s\rightarrow (x, e)},
\frac{(e, s)\rightarrow (x, e')\ (f, s) \rightarrow v\ (e', s[x\mapsto v])\rightarrow v'}{e\ f,s\rightarrow v'}$$
\paragraph{Static binding with lazy evaluation}
Is lazy (call by name) even different than eager (call by value)?
Suppose $ e $ is an expression that does not terminate.
Find $ f $ that uses $ e $ such that its lazy semantics are different than eager.
$$ (\lambda x.5)\ e $$
In lazy, the value is 5. In eager, it does not terminate.
This is an example of a side effect: not pure function.
$$
\begin{array}{c|c|c}
& \text{STATIC}& \text{DYNAMIC}\\\hline
	\text{EAGER} &
	\begin{matrix}
		Val=\mathbb{Z}\cup (Var\times Expr\times St)\\
		St=Var\rightarrow Val\\
		\text{mutually recursive}
	\end{matrix} &
	\begin{matrix}
		Val=\mathbb{Z}\cup (Var\times Expr)\\
		St=Var\rightarrow Val\\
		\text{not recursive anymore!}
	\end{matrix}\\\hline
\text{LAZY} &
	\begin{matrix}
		Val=\mathbb{Z}\cup (Var\times Expr\times St)\\
		St=Var\rightarrow (Expr \times St)\\
		\text{just state is recursive}
	\end{matrix} &
\end{array}
$$
Now the big step semantics:
$$\frac{}{n,s\rightarrow n}\frac{s(x)=(e,s')\ (e,s')\rightarrow v}{x,s\rightarrow v}
\frac{e,s\rightarrow\underline{m}, f,s\rightarrow \underline{n}}{e+f,s\rightarrow m+n}$$

$$\frac{}{\lambda x.e,s\rightarrow (x, e,s)},
\frac{e, s\rightarrow (x, e', s')\ , (e's'[x\mapsto (f,s)])\rightarrow v}{e\ f,s\rightarrow v}$$
We do not evaluate $ f $ anymore, we just pass it inside $ e' $!
\paragraph{Dynamic binding with lazy evaluation}
Example for a difference between static and dynamic under lazy evaluation:
$$(\lambda x(\lambda y\lambda x\ y)x\ 3)5$$
Under static we get 5, because $ y $ gets bound to expression $ x $, with the environment where $ x $ was bound to 5, while in static the expression $x$ gets evaluated in the internal environment where $ x $ is bound to 3.

$$
\begin{array}{c|c|c}
& \text{STATIC}& \text{DYNAMIC}\\\hline
\text{EAGER} &
	\begin{matrix}
		Val=\mathbb{Z}\cup (Var\times Expr\times St)\\
		St=Var\rightarrow Val\\
		\text{mutually recursive}
	\end{matrix} &
	\begin{matrix}
		Val=\mathbb{Z}\cup (Var\times Expr)\\
		St=Var\rightarrow Val\\
		\text{not recursive anymore!}
	\end{matrix}\\\hline
\text{LAZY} &
	\begin{matrix}
		Val=\mathbb{Z}\cup (Var\times Expr\times St)\\
		St=Var\rightarrow (Expr \times St)\\
		\text{just state is recursive}
	\end{matrix} &
	\begin{matrix}
		Val=\mathbb{Z}\cup (Var\times Expr)\\
		St=Var\rightarrow Expr\\
		\text{again, not recursive!}
	\end{matrix}
\end{array}
$$
And the rules:
$$
\frac{s(x)=e\ (e,s)\rightarrow v}{x,s\rightarrow v}
\frac{}{\lambda x.e,s\rightarrow (x, e)}$$$$
\frac{(e, s)\rightarrow (x, e')\ (e', s[x\mapsto f])\rightarrow v}{e\ f,s\rightarrow v}$$
\section{Ćwiczenia 21/11}
$$c::=\dots|\texttt{for }x=e\texttt{ to } f\texttt{ try } c\texttt{ else } d|\texttt{fail}$$
How to interpret this?
\begin{enumerate}
	\item If $ e>f $ then do $ d $.
	\item Otherwise, $ x:=e $.
	\item Do $ c $.
	\item If $ c $ succeeds, then succeed and restore x.
	\item If $ c $ fails, then $ x:=x+1 $.
	\item If $ x\le n  $ go to step 3
	\item Otherwise restore $ x $ and succeed
\end{enumerate}
\paragraph{Big step semantics}
$$C:=c\times St\cup St\times\{\top, \bot\}\text{success - }\top,\text{ fail - }\bot$$
$$St=Var\rightarrow\mathbb{N}$$

$$\frac{}{\texttt{skip}, s\rightarrow s,\top}$$
$$\frac{}{\texttt{fail}, s\rightarrow s, \bot}$$
$$\frac{
	c,s\rightarrow s',\top\ \ \ 
	d,s'\rightarrow v
}{
	c;d, s\rightarrow v
}$$
$$\frac{c,s\rightarrow s',\bot}{
	c;d, s\rightarrow s',\bot
}$$

$$\frac{
	e,s\rightarrow m\ \ \ 
	f,s\rightarrow n\ \ \
	m\le n\ \ \ 
	c,s[x\mapsto m]\rightarrow s',\top
}{
	\texttt{for }x=e\texttt{ to }f\texttt{ try }
	c\texttt{ else }d, s\rightarrow s'[x\mapsto s(x)],\top
}$$

$$\frac{
	e,s\rightarrow m\ \ \ 
	f,s\rightarrow n\ \ \
	m > n\ \ \ 
	d,s\rightarrow v
}{
	\texttt{for }x=e\texttt{ to }f\texttt{ try }
	c\texttt{ else }d, s\rightarrow v
}$$

$$\frac{
	e,s\rightarrow m\ \ \ 
	f,s\rightarrow n\ \ \
	m\le n\ \ \ 
	c,s[x\mapsto m]\rightarrow s',\bot\ \ \ 
	\texttt{for }x=m+1\texttt{ to }n\texttt{ try }
	c\texttt{ else skip}, s'\rightarrow s'', \_
}{
	\texttt{for }x=e\texttt{ to }f\texttt{ try }
	c\texttt{ else }d, s\rightarrow s'[x\mapsto s(x)],\top
}$$


\subsection{Exceptions}
$$c::=\dots|\texttt{throw}(e)|\texttt{try } c \texttt{ catch }(e)d$$
\paragraph{Configurations}
$$C:=c\times St\cup St\times(\mathbb{N}\cup\{\top\})$$
$$\frac{}{\texttt{skip}, s\rightarrow s,\top}$$
$$\frac{e\rightarrow k}{\texttt{throw }(e), s\rightarrow s,k}$$
$$\frac{
	c,s\rightarrow s',\top\ \ \ 
	d,s'\rightarrow v
}{
	c;d, s\rightarrow v
}$$
$$\frac{
	c,s\rightarrow (s',k), k\not=\top
}{
	c;d, s\rightarrow s',k
}$$
$$\frac{
	c\rightarrow s',\top
}{
	\texttt{try } c \texttt{ catch }(e)\ d, s\rightarrow s',\top
}$$
$$\frac{
	c\rightarrow (s',k), k\not=\top\ \ \ 
	e,s'\rightarrow n\ \ \ 
	n=k\ \ \ 
	d,s'\rightarrow v
}{
	\texttt{try } c \texttt{ catch }(e)\ d, s\rightarrow v
}$$
$$\frac{
	c\rightarrow (s',k), k\not=\top\ \ \ 
	e,s'\rightarrow n\ \ \ 
	n\not=k\ \ \ 
	d,s'\rightarrow v
}{
	\texttt{try } c \texttt{ catch }(e)\ d, s\rightarrow s',k
}$$

\subsection{Nondeterministic programming language}
$$c::=1|c+c|c-c|x:=c|c;c|c\texttt{ or } c|c?|c*$$
\begin{itemize}
\item assignment returns $c$
\item semicolon returns second command
\item \texttt{or} executes non-deterministically and returns what it executed
\item question mark returns 1 if $c = 1$ and has no semantics otherwise
\item star - non-deterministically choose a natural number and execute $c$ that many times
\end{itemize}
Execute b, if it's 1, calculate c and repeat.
$$\texttt{while }b\texttt{ do } c\equiv (b?;c)*;(1-b)?$$
We forced the program to non-deterministically choose the right amount of iterations in *.
$$ \texttt{if }b\texttt{ then }c\texttt{ else }d\equiv (b?;c)\texttt{ or }((1-b)?;d) $$
\paragraph{Big step semantics}
$$ C:=c\times St\cup St\times\mathbb{Z} $$
$$\frac{}{1\rightarrow s,1}$$
$$\frac{
	c,s\rightarrow s',m\ \ \ 
	d,s'\rightarrow s'',n
}{c+d\rightarrow s'',m+n}$$
$$\frac{
	c,s\rightarrow s',n\ \ \ 
}{x:=c\rightarrow s'[x\mapsto n],n}$$
$$\frac{
	c,s\rightarrow s',n\ \ \ 
	d,s'\rightarrow s'',m
}{c;d\rightarrow s'',m}$$
$$\frac{
	c,s\rightarrow s',n
}{c\texttt{ or }d\rightarrow s',m}$$
$$\frac{
	d,s\rightarrow s',n
}{c\texttt{ or }d\rightarrow s',n}$$
$$\frac{
	c,s\rightarrow s',1
}{c?\rightarrow s',1}$$
$$c*\equiv (c*;c) or 1$$


\section{Tutorial 28/11}
\subsection{Local blocks and variables}
$$C\ni c::=\dots\ |\ \texttt{begin }\{d\}\texttt{ in } c$$
$$D\ni d::=\texttt{Var }x|d;d$$
We don't have state anymore, instead we get:
$$s\in Store = Loc  \rightarrow \mathbb{N}$$
$$\rho\in Env = Var \rightarrow Loc$$
Loc is a countable set of memory locations. We need not 
To get environments, we need a function $ newloc:Env \rightarrow Loc $ that satisfies $ newloc(\rho)\not\in dom(\rho) $.\\
Configurations:
$$ C_1=C\times Store\times Env \cup Store 
\text{ for statements}$$
$$ C_2= D\times Env\cup Env
\text{ for declarations}$$

Expressions are just like before, except:
$$\frac{}{x,s,\rho \rightarrow s(\rho(x))}$$

Statements:
$$\frac{
	c_1,s,\rho \rightarrow s'\ \ \ 
	c_2,s',\rho \rightarrow s''
}{c_1;c_2,s,\rho \rightarrow s''}$$
$$\frac{
	e,s,\rho \rightarrow n
}{
	x:=e,s,p \rightarrow s[\rho(x)\mapsto n]
}$$
$$\frac{
	d,\rho \rightarrow \rho'\ \ \ 
	c,s,\rho' \rightarrow s'
}{
	\texttt{begin }\{d\}\texttt{ in } c,s,p \rightarrow s'
}$$
Declarations:
$$\frac{
}{
	\texttt{Var }x,\rho \rightarrow \rho[x\mapsto newloc(\rho)]
}$$
$$\frac{
	d_1,\rho  \rightarrow \rho'\ \ \ 
	d_2,\rho'  \rightarrow \rho''
}{
	d_1;d_2,\rho \rightarrow \rho''
}$$
\subsection{Procedures with one integer variable parameter}
We extend the declaration syntax:
$$d::=\texttt{Var }x\ |\ d;d\ |\ \texttt{proc }x(y):=c$$
Where $ x $ is the procedure name and $ y $ is a formal parameter.
We also extend the statement syntax:
$$C\ni c::=\dots\ |\ \texttt{begin }\{d\}\texttt{ in } c\ |\ \texttt{call }x(y)$$
where $ x $ is again the procedure name and $ y $ is the actual parameter.
\subsection{Static binding with eager evaluation}
We need to extend the environment to be able to store procedures.
We'll use closures consisting of argument, command and environment.
$$Cl=Var\times C\times Env$$
The environment will also be different:
$$\rho\in Env = Var \rightharpoonup (Loc\cup Cl)$$
Big steps:
$$\frac{
	\rho(x)=(z,c,\rho')\ \ \ 
	l = newloc(\rho')\ \ \ 
	c,s[l\mapsto s(\rho(y))],\rho'[z\mapsto l] \rightarrow s'
}{\texttt{call }x(y),s,\rho \rightarrow s'}$$
$$\frac{
	cl=(y,c,\rho)
}{\texttt{proc }x(y):=c,\rho \rightarrow \rho[x\mapsto cl]}$$
\subsubsection{Is this \st{loss} recursion?}
$ \texttt{begin }\{\texttt{Var }z;\texttt{proc }x(y):=(z:=y);\texttt{proc }x(y):=\texttt{call }x(y)\} \texttt{call }x(y) $
The second procedure calls the first one!
\subsubsection{Are we recursion yet?}
Rewrite the procedure decl semantics:
$$\frac{
	cl=(y,c,\rho[x\mapsto cl])
}{\texttt{proc }x(y):=c,\rho \rightarrow \rho[x\mapsto cl]}$$
This is an infinite closure!!!1!\\
How do we patch this? Either allow infinite closures (XD) or patch this at calltime.
$$\frac{
	\rho(x)=(z,c,\rho')\ \ \ 
	l = newloc(\rho')\ \ \ 
	c,s[l\mapsto s(\rho(y))],\rho'[z\mapsto l][x\mapsto(z,c,\rho')] \rightarrow s'
}{\texttt{call }x(y),s,\rho \rightarrow s'}$$
There we update $ \rho' $ by mapping to $ \rho' $! My lord, is this legal? We will make it legal! This is just a definition of $ \rho'' = \rho'[z\mapsto l][x\mapsto(z,c,\rho')] $ which happens to rely on $\rho'$ twice! Everything is cool.
\subsection{Dynamic binding with eager evaluation}
Now closures don't record environment!
$$Cl=Var\times C$$
$$\frac{
	\rho(x)=(z,c)\ \ \ 
	l = newloc(\rho)\ \ \ 
	c,s[l\mapsto s(\rho(y))],\rho[z\mapsto l] \rightarrow s'
}{\texttt{call }x(y),s,\rho \rightarrow s'}$$
$$\frac{
	cl=(y,c)
}{\texttt{proc }x(y):=c,\rho \rightarrow \rho[x\mapsto cl]}$$
Because we do the \texttt{call} with the environment local to the caller, the caller already has defined the function that they call, so we get recursion for freeeee!
\subsection{Call by need}
This tries to combine call by name and call by value.
Call by name:
\begin{itemize}
	\item terminates more often - good
	\item evaluates programs only if they are needed - good
	\item but if we need an argument many times, it gets evaluated again! - bad
	\item this is C macros in a nutshell
\end{itemize}
Call by value:
\begin{itemize}
	\item We evaluate every argument \textbf{exactly} once - good and bad
	\item good bc we don't reevaluate
	\item bad bc we can evaluate unnecessarily
\end{itemize}
Call by need combines them into the ultimate Haskell experience:
$$Store=Loc \rightarrow (\mathbb{N}\cup Expr)$$
$$Env=Var \rightarrow (Loc\times Cl)$$
$$Cl=Var\times C$$
$$Conf=C_0\cup C_1\cup C_2$$
$C_0$ is for expressions, $ C_1 $ and $ C_2 $ are the same as before.
$$C_0=Expr\times Store\times Env\cup \mathbb{N}\times Store$$
$$\frac{s(\rho(x))=n}{x,s,p \rightarrow n, s}$$
$$\frac{
	s(\rho(x))=e\ \ \ 
	e,s,\rho \rightarrow n
}{
	x,s,p \rightarrow n,s[\rho(x)\mapsto n]
}$$
$$\frac{
	\rho(x)=(y,c)\ \ \ 
	l=newloc(\rho)\ \ \ 
	c,s[l\mapsto e],\rho[y\mapsto l] \rightarrow s'
}{
	\texttt{call }x(e),s,p \rightarrow n,s[\rho(x)\mapsto n]
}$$

\section{Tutorial 5/12}
\subsection{Denotational semantics}
\subsubsection{Arithmetic expressions}
$$[[e]]_s\in\mathbb{N}, [[\_]]\_:Expr \rightarrow State \rightarrow \mathbb{N}, State: Var \rightarrow \mathbb{N}$$
$$[[n]]s=n \text{, sometimes we also write }[[n]]=\lambda s.n $$
This is the same, but not in set theory without extentionality axiom.
$$[[x]]s=s\ x\ \ \ [[x]]=\lambda s.s\ x \text{ - apply}$$
$$[[e+f]]s=[[e]]_s+[[f]]s$$
$$[[\texttt{let }x=e\texttt{ in }f]]s=[[f]]{s[x\mapsto[[e]]s]}$$
\subsubsection{Programs}
$$[[c]]s\in State_\bot, [[\_]]\_:C \rightarrow States \rightarrow States_\bot, States_\bot=States\cup\{\bot\}$$
$$[[\texttt{skip}]]s=s \text{ - pointful notation, }[[\texttt{skip}]]=\lambda s.s=id \text{ - point-free (point-less) notation}$$
$$[[x:=e]]s=s[x\mapsto[[e]]s]$$
$$[[c;d]]s=[[d]]([[c]]s)$$
Introducing strictness in the semantic function: $[[c]]\bot=\bot$ regardless of what $ c $ is.
$$[[c;d]]=\lambda s.[[d]]([[c]]s)=[[d]]\circ[[c]]$$
For \texttt{if then else} we'll use a helper function $ Cond_D(x, d, e)=\bigg\{\begin{matrix}
d&\texttt{if $ x =$ true}\\
e&\texttt{if $ x =$ false}
\end{matrix} $ where $ d, e\in D $ and $ x\in\{\texttt{true, false}\} $
$$[[\texttt{if $ b $ then $ c $ else $ d $}]]s=Cond_{State}([[b]]s,[[c]]s,[[d]]s)$$
For our denotational semantics, if we want it to uniquely define a semantic we need compositionality.
$$[[\texttt{while $ b $ do $ c $}]]s=[[\texttt{if $ b $ then $(c;$while $ b $ do $ c) $ else skip}]]s =$$$$=Cond([[b]]s[[\texttt{while $ b $ do $ c $}]]([[c]]s), s)$$
This is, again, not a definition! 
If $w=\texttt{while $ b $ do $ c $}$, we have the following:
$$[[w]]s=[[w]]([[\texttt{skip}]]s)=[[w]]s$$
This means that our 'rule' didn't give us anything! How to do it?
Let's define a partial order on states; $ \forall_{s\in States_\bot} \bot\sqsubseteq S, S\sqsubseteq S$. We'll extend that to functions:
$$f,g : States_\bot \rightarrow States_\bot f\sqsubseteq g \text{ iff } \forall s\ f\ s\sqsubseteq g\ s$$
Note that $ f,g\sim[[c]] $.\\
If an operator $ \Phi : (States \rightarrow States_\bot) \rightarrow (States \rightarrow States_\bot)$ that applies the $ Cond $ function ($ \Phi(f)=\lambda s.Cond([[b]]s, f([[c]]s)),s $) has fixed points, that is $ f=\Phi\ f $, then we define the semantics of $ w $ as the least fixed point: $ [[w]]=\mu f.\Phi(f) $

\section{Tutorial 12/12}
\subsection{Denotational semantics of local variables and procedures}
\subsubsection{Call-by-name with static binding}
Syntax:
$$c::=\dots | \texttt{begin \{$d$\} in } c |\texttt{ call }x(y)$$
$$d::=\texttt{var }x:=e|\texttt{proc x($y$) := }c | d,d$$
Semantic domains:
$$Loc=\{0,1,\dots\}$$
$$s\in Store = Loc \rightarrow \mathbb{Z}$$
$$\rho\in Env = Var \rightarrow Loc$$
$$\pi\in PEnv = Var \rightarrow Proc$$
In operational we used syntax and semantics to create closures. Here we can do better.
In call by value, we'd use
$$Proc = \mathbb{Z} \rightarrow Store_\bot \rightarrow Store_\bot$$
Let's try that first.
$$C[[c]]:(Env\times PEnv \times Store_\bot) \rightarrow Store_\bot$$
$$D[[d]]:(Env\times PEnv \times Store) \rightarrow (Env\times PEnv \times Store)$$
Rules:
$$C[[\texttt{skip}]](\rho, \pi, s)=s$$
$$C[[x:=e]](\rho, \pi, s)=s[\rho\ x\mapsto E[[e]]\rho s]$$
$$C[[C_1;C_2]](\rho, \pi, s)=C[[C_2]](\rho, \pi, C[[C_1]](\rho,\pi, s))$$
$$C[[\texttt{while $ b $ do $ c $}]](\rho, \pi, s)=$$
We need to find a fixed point of $ Cond:Bool \rightarrow Store_\bot \rightarrow Store_\bot \rightarrow Store_\bot $ again!
$$C[[\texttt{while $ b $ do $ c $}]](\rho, \pi, s)=Cond(B[[b]]\rho s,C[[w]](\rho, \pi, C[[c]]\rho\pi s), s)$$
This is just an equation, not a definition!
$$\Phi(F)=\lambda s'.Cond(B[[b]]\rho s',F(\rho, \pi, C[[c]]\rho\pi s'), s')$$
$$C[[\texttt{while $ b $ do $ c $}]](\rho, \pi, s)=(\mu F.\Phi)s$$
($ \mu x.y $ means lowest $x$ that is a fixed point of $y$)
$$C[[\texttt{begin \{$ d $\} in } c]](\rho, \pi, s)=C[[c]](D[[d]](\rho, \pi, s))=C[[c]]$$
$$C[[\texttt{call x($ y $)}]](\rho, \pi, s)=\pi(x)s(\rho(y))s$$

$$D[[\texttt{var }x:=e]](\rho, \pi, s)=(\rho[x\mapsto e],\pi,s[l\mapsto[[e]]\rho s])$$
$$D[[\texttt{proc }x(y):=c]](\rho, \pi,s)=(\rho, \pi[x\mapsto \lambda n.\lambda s'.C[[c]](\rho[y\mapsto l], s'[l\mapsto n])], s)$$
Where  $l=newloc(\rho)$.

\paragraph{Now call by name}
$$Proc = Loc \rightarrow Store_\bot \rightarrow Store_\bot$$
$$D[[\texttt{proc }x(y):=c]](\rho, \pi,s)=(\rho, \pi[x\mapsto \lambda v.\lambda s'.C[[c]](\rho[y\mapsto v], \pi,s')], s)$$
$$C[[\texttt{call x($ y $)}]](\rho, \pi, s)=\pi(x)(\rho\ y)s$$

\subsection{Continuations}
Let's define a function:
$$f\ 0 = 1$$
$$f\ n=nf(n-1)$$
This is a classic style factorial $ f:\mathbb{N} \rightarrow \mathbb{N^+}$.
The continuation style factorial is $ \hat{f}:\mathbb{N} \rightarrow (\mathbb{N^+} \rightarrow \mathbb{N^+}) \rightarrow \mathbb{N^+} $
$$\hat{f}\ 0\ k = k\ 1$$
$$\hat{f}\ n\ k = \hat{f} (n - 1)(\lambda m.k(nm))$$

Let's prove that the produce the same results, i.e. $ f\ n=\hat{f}\ n\ id$:
First, let $ n*\_ =\lambda m.n*m$, so that 
$$\hat{f}\ n\ k = \hat{f} (n - 1)(\lambda m.k(nm))=\hat{f}(n-1)(k\circ (n*\_))$$

Easier to see if defined as:
$$\hat{f}\ 0=\lambda k.k\ 1$$
$$\hat{f}\ n=\lambda k.\hat{f}(n-1)(k\circ(n*\_))\ 1$$
\section{Tutorial 19/12}
\subsection{Fibonacci with continuations}
Direct:
$$f\ 0=0$$
$$f\ 1=1$$
$$f\ n=f(n-1)+f(n-2)$$
Continuation style:
$$f'\ k\ 0=k\ 0$$
$$f'\ k\ 1=k\ 1$$
$$f'\ k\ n=f'\ (\lambda a.f'\ (\lambda b.k(a+b))(n-2))(n-1)$$

\subsection{Denotational semantics of programs with continuations}
Let $ Cont_A = A \rightarrow A $
Direct style was:
$$C[[c]]:State_\bot \rightarrow State_\bot$$
And now we want:
$$C[[c]]:Cont_{State_\bot} \rightarrow State_\bot \rightarrow State_\bot$$

Semantics of commands:
$$C[[\texttt{skip}]]=id$$
$$C[[x:=e]]k=\lambda s.k\ s[x:=E[[e]]s]$$
$$C[[c;d]]k=C[[c]](C[[d]]k)$$
also
$$C[[c;d]]=C[[c]]\circ C[[d]]$$
Note that direct style is the same, but in the opposite order.
$$C[[\texttt{if $ b $ then $ c $ else $ d $}]]k=\lambda s. Cond(B[[b]]s, C[[c]]k\ s, C[[d]]k\ s)$$
Let $ w=\texttt{while $b$ do $c$} $
$$C[[w]]k=\lambda s.Cond(B[[b]]s, C[[c]](C[[w]]k)s,k\ s) $$
Again, we need to do tha fixed point.
$$\Phi(F)=\lambda\beta.\lambda s.Cond([[b]]s,C[[s]](F\beta)s, \beta s)$$
And the real semantics of $ w $ is the least fixed point of $ \Phi $.
\subsection{other stuff}
Abort:
$$ C[[\texttt{abort}]]k=id_{State_\bot} $$
$$ C[[\texttt{break}]]k=?$$
We need to change the semantics:
$$C[[c]]:Cont_{State_\bot} \rightarrow Cont_{State_\bot} \rightarrow State_\bot \rightarrow State_\bot$$
And rewrite all rules!
$$C[[\texttt{skip}]]k\alpha=k$$
$$ C[[\texttt{abort}]]k\alpha=id_{State_\bot} $$
$$C[[\texttt{break}]]k\alpha=\alpha$$
$$C[[x:=e]]k\alpha=\lambda s.k\ s[x:=E[[e]]s]$$
$$C[[c;d]]k=C[[c]](C[[d]]k\alpha)\alpha$$
$$C[[\texttt{while $ b $ do $ c $}]]k\alpha=(\mu F\Phi(F))k$$
and 
$$\Phi(F)=\lambda\beta.\lambda s.Cond([[b]]s,C[[s]](F\beta)\beta s, \beta s)$$
\subsection{Exceptions}
We need a map, continuation environment 
$ CEnv=\mathbb{Z} \rightarrow Cont_{State_\bot} $
$$C[[c]]:Cont_{State_\bot} \rightarrow Cont_{State_\bot}  \rightarrow CEnv \rightarrow State_\bot \rightarrow State_\bot$$
$$C[[\text{throw } e]]k\alpha\rho=\lambda s.\rho(E[[e]]s)s$$
$$C[[\text{try $ c $ catch $ e\ d $}]]k\alpha\rho=\lambda s.([[c]]k\alpha\rho[[[e]]s\mapsto(C[[d]]k\alpha\rho)])s$$
\subsection{Arithmetic expressions - continuation style}
Direct was $ E[[e]]:State_\bot \rightarrow \mathbb{Z} $, continuation will be: $ E[[e]] : ECont \rightarrow State_\bot \rightarrow State_\bot$, where $\alpha\in ECont=\mathbb{Z} \rightarrow State_\bot $.
With that, let's do assignment first:
$$C[[x:=e]]k=\lambda s.E[[e]](\lambda n.k\ s[x\mapsto n])s$$
$$ E[[n]]\alpha =\lambda s.\alpha\ n$$
$$ E[[x]]\alpha =\lambda s.\alpha\ (s\ x)$$

$$ E[[e+f]]\alpha =\lambda s.E[[e]](\lambda m.E[[f]](\lambda n.\alpha(n+m))s)s$$
$ m  $ is the value of $ e $
This also encodes the order of evaluation of subexpressions.
\section{Tutorial 9/01}
\subsection{Axiomatic semantics}
Hoare triplet:
$$\{A\}c\{B\}$$
When we show that $A$ is true, then we run $c$ and if it terminates, then $ B $ holds, then we can say that the triple is true.\\
$ A, B $ are formulas over $ (\mathbb{Z}, +, \cdot, 0, 1, \le) $\\
The semantics of a formula is the set of tuples that satisfy the formula.
Now we have rules from which we can derive provable triplets.
$$\frac{}{\{A\}\texttt{skip}\{A\}}$$
$$\frac{
	\{A\}c\{C\}\ \ \ 
	\{C\}d\{B\}
}{
	\{A\}c;d\{B\}
}$$
$$\frac{
	\{A\wedge b\}c\{B\}\ \ \ 
	\{A\wedge \neg b\}d\{B\}
}{
	\{A\}\texttt{if $b$ then $c$ else $d$}\{B\}
}$$
For assignment, we need to reason backwards. We look at what we get after the assignment, and replace all occurrences of the variable we assign to with the expression we assign:
$$\frac{}{\{A[x\mapsto e]\}x:=e\{A\}}$$
Now the last rule is the while loop:
$$\frac{
	\{A\wedge b\}c\{A\}
}{
	\{A\}\texttt{while $ b $ do $ c $}\{A \wedge \neg b\}
}$$
And the weakening role
$$\frac{
	A\implies A'\ \ \ 
	\{A'\}c\{B'\}\ \ \ 
	B'\implies B
}{
	\{A\}c\{B\}
}$$
This is not complete (everything true can be proven), but is relatively complete relative to first order logic, because we need it to prove the two implications above.
\subsection{Examples}
$$\{\texttt{true}\}c\{\texttt{false}\}$$
This triple is true if and only if $ c $ does not terminate
\begin{lstlisting}[escapeinside={(*}{*)}]
(*$\{n\ge1\}$*)
x := 1;(*$\{x=1\wedge n\ge 1\}$*)
y := 0;(*$\{x=1\wedge y=0\wedge n\ge 1\} = A$*)
while (*$x\le n$*) do (y := y + x; x := x + 1)
(*$\{2y=n(n+1)\} = B$*)
\end{lstlisting}
Now we need to find a loop invariant $I$ that is implied by $A$ and $I\wedge \neg x\le n$ implies $ B $.
A good candidate is $ y=\frac{x(x-1)}{2} $, but this doesn't exactly imply $ B $, just puts a bound on y, we also need to say that $ x\le n+1 $, which combined with what we had implies $ B $.
\end{document}